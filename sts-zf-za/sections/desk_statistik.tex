\subsection{Wichtige Eigenschaften der Kovarianz und Korrelation}
\label{sec:desk_statistik_kovarianz_korrelation}
\begin{enumerate}
    \item \tilde{s}_{xy} und r_{xy} messen den Zusammenhang der beiden Merkmale, 
        d.h. wie nahe die  Datenpunkte einer Geraden mit der Gleichung $y = m + q$ mit $m \neq 0$ kommen.
    \item $-1 \leq r_{xy} \leq 1$.
    \item $r_{xy} = 0$, bzw. $\tilde{s}_{xy} = 0$ oder nahe bei Null, bedeutet, 
        dass die Punkte $(x_1, y_1), \dots, (x_n,y_n)$ gleichmässig um den Schwerpunkt 
        $(\bar{x}, \bar{y})$ verteilt sind.
    \item $r_{xy} = 1$ bedeutet, dass ein positiver linearer Zusammenhang zwischen den Merkmalen besteht.
    \item $r_{xy} = -1$ bedeutet, dass ein negativer linearer Zusammenhang zwischen den Merkmalen besteht.
    \item $r_{xy} > 0$ bedeutet, dass die Punkte tendenziell um eine Gerade mit positiver Steigung 
        $(m > 0)$ verteilt sind (Gleichsinniger linearer Zusammenhang, positive Korrelation).
    \item $r_{xy} < 0$ bedeutet, dass die Punkte tendenziell um eine Gerade mit negativer Steigung
        $(m < 0)$ verteilt sind (Gleichsinniger linearer Zusammenhang, negative Korrelation).
    \item $r_{xy}$ ist nicht robust, d.h. ein Ausreisser kann den Wert von $r_{xy}$ stark verändern.
    \item $\tilde{s}_{xy} = \bar{xy} - \bar{x} \cdot \bar{y}$
    \item $r_{xy} = \frac{\bar{xy} - \bar{x} \cdot \bar{y}}{\sqrt{(\bar{x^2} - \bar{x}^2) \cdot (\bar{y^2} - \bar{y}^2)}}$
\end{enumerate}

\subsection{Rangkorrelation}
\label{sec:desk_statistik_rangkorrelation}
Sei $(x_1, \dots, x_n)$ eine Stichprobe eines metrischen Merkmals. \\
Der Rang $rg(x_i)$ eines Stichprobenwertes $x_i$ ist definiert als der Index von $x_i$ 
in der nach der Grösse sortierten Stichprobe, wenn die Stichprobe keine Wiederholungen enthält. \\
Tritt der Stichprobenwert $x_i$ mehrfach auf, so wird der Rang $rg(x_i)$ durch die Formel
\begin{equation*}
    rg(x_i) = \frac{1}{k} \cdot \sum_{j=1}^k j
\end{equation*}
berechnet, wobei $k$ die Anzahl der Wiederholungen von $x_i$ ist. \\

\subsubsection{Rangkorrelation, Definition}
\label{sec:desk_statistik_rangkorrelation_zwei_merkmale}
Wir setzen eine bivariate Stichprobe $(x_1, y_1), \dots, (x_n, y_n)$ der Länge $n$ voraus. \\
Dazu bilden wir die zugehörigen Rangfolgen:
\begin{enumerate}
    \item Die Folge der Rangpaare: $rg(xy) = ((rg(x_i), rg(y_i)), \dots, (rg(x_n), rg(y_n)))$
    \item Die Folge der Ränge der $x$-Werte: $rg(x) = (rg(x_1), \dots, rg(x_n))$
    \item Die Folge der Ränge der $y$-Werte: $rg(y) = (rg(y_1), \dots, rg(y_n))$
\end{enumerate}
Der (empirische) Korrelationskoeffizient nach Spearman (Rangkorrelationskoeffizient) 
ist definiert als der Pearson Korrelationskoeffizient der Rangfolgen $r_{Sp} = r_{rg(xy)}$.

\subsection{Wichtige Eigenschaften der Rangkorrelation}
\label{sec:desk_statistik_rangkorrelation_eigenschaften}
\begin{enumerate}
    \item $\tilde{s}_{rg(xy)}$ und $r_{Sp}$ messen den monotonen Zusammenhang der beiden Merkmale, 
        d.h. wie nahe die Datenpunkte einer streng monotonen Funktion kommen. In anderen Worten: Es wird
        gemessen, wie gut die Rangordnungen in den $x$ und $y$-Werten sich entsprechen.
    \item $-1 \leq r_{Sp} \leq 1$.
    \item $r_{Sp} = 0$, bzw. $\tilde{s}_{rg(xy)} = 0$, bedeutet, 
        dass die Punkte $(x_1, y_1), \dots, (x_n,y_n)$ gleichmässig um den Schwerpunkt der Ränge
        $(\bar{rg(x)}, \bar{rg(y)}) = (\frac{n+1}{2}, \frac{n+1}{2})$ verteilt sind.
    \item $r_{Sp} = 1$ bedeutet, dass ein streng monoton wachsender funktionaler Zusammenhang 
        zwischen den Merkmalen besteht.
    \item $r_{Sp} = -1$ bedeutet, dass ein streng monoton fallender funktionaler Zusammenhang
        zwischen den Merkmalen besteht.
    \item $r_{Sp}$ ist robust, d.h. ein Ausreisser hat keinen grossen Einfluss auf den Wert.
    \item $\tilde{s}_{rg(xy)} = \bar{rg(xy)} - \bar{rg(x)} \cdot \bar{rg(y)} = \bar{rg(xy)} - \frac{(n+1)^2}{4}$
    \item $r_{Sp} = \frac{\bar{rg(xy)} - \bar{rg(xy)} \cdot \bar{rg(y)}}{\sqrt{\bar{rg(x)^2} - \bar{rg(x)}^2 \cdot \bar{rg(y)^2} - \bar{rg(y)}^2}} 
        = \frac{\bar{rg(xy)} - \frac{(n+1)^2}{4}}{\sqrt{\bar{rg((x)^2)} - \frac{(n+1)^2}{4} \cdot \sqrt{\bar{rg(y^2)}} - \frac{(n+1)^2}{4}}}$
    \item Falls jeder Stichprobenwert der $x$ und der $y$-Werte nur einmal vorkommt, mit anderen Worten, falls die Ränge in den 
        $x$ und $y$-Werten alle verschieden sind (also keine Bindungen auftreten), gilt: $r_{Sp} = 1 - \frac{6 \cdot \sum_{i=1}^n  d_i^2}{n \cdot (n^2 - 1)}$
        mit $d_i = rg(y_i) - rg(x_i)$.